% Template for a letter on IAG-USP letterhead

%%%%%%%%%%%%%%%%%%%%%%%%%%%%%%%%%%%%%%%%%%%%%%%%%%%%%%%%%%%%%%%%%%%%%%%%%%%%%%%
% Set a class and general configuration
\documentclass[a4paper,onecolumn,10pt]{article}

%%%%%%%%%%%%%%%%%%%%%%%%%%%%%%%%%%%%%%%%%%%%%%%%%%%%%%%%%%%%%%%%%%%%%%%%%%%%%%%
% Set variables with the title, authors, etc.
\newcommand{\Titulo}{Carta de motivação - Embaixadores da Rede Brasileira de Reprodutibilidade}
\newcommand{\Nome}{Leonardo Uieda}
\newcommand{\Cargo}{Professor Doutor}
\newcommand{\Email}{uieda@usp.br}
\newcommand{\Telefone}{+55 11 30914787}

%%%%%%%%%%%%%%%%%%%%%%%%%%%%%%%%%%%%%%%%%%%%%%%%%%%%%%%%%%%%%%%%%%%%%%%%%%%%%%%
% Import the required packages
\usepackage[utf8]{inputenc}
\usepackage[TU]{fontenc}
\usepackage[brazil]{babel}
%\usepackage[english]{babel}
\usepackage{graphicx}
\usepackage{hyperref}
\usepackage{fancyhdr}
\usepackage{geometry}
\usepackage{microtype}
\usepackage{xcolor}
% improved urls with proper hyphenation
\usepackage{xurl}
% Use a different font
\usepackage[default,scale=0.95]{opensans}
% Icons and fonts (requires using xelatex or luatex)
\usepackage{fontawesome5}
\usepackage{academicons}
% Control the font size
\usepackage{anyfontsize}
\usepackage{setspace}
% Generate random text
\usepackage{lipsum}
% Better left and right align
\usepackage{ragged2e}

%%%%%%%%%%%%%%%%%%%%%%%%%%%%%%%%%%%%%%%%%%%%%%%%%%%%%%%%%%%%%%%%%%%%%%%%%%%%%%%
% Configuration of the document

\geometry{%
  left=25mm,
  right=15mm,
  top=15mm,
  bottom=15mm,
  headsep=15mm,
  headheight=15mm,
  footskip=7mm,
  includehead=true,
  includefoot=true
}

% Control line spacing
\onehalfspacing
\newcommand{\Padding}{\vspace{0.5cm}}

% Custom colors
\definecolor{darkgray}{gray}{0.4}
\definecolor{mediumgray}{gray}{0.5}
\definecolor{lightgray}{gray}{0.9}
\definecolor{mediumblue}{HTML}{2060c2}
\definecolor{lightblue}{HTML}{f7faff}

% Make urls use the same font as every other text
\urlstyle{same}

% Configure hyperref and add PDF metadata
\hypersetup{
    colorlinks,
    allcolors=mediumblue,
    pdftitle={\Titulo},
    pdfauthor={\Nome},
    breaklinks=true,
}

% Configure header and footer
\newcommand{\HeaderFont}{\footnotesize\color{mediumgray}}
\pagestyle{fancy}
\fancyhf{}
\rhead{%
  \includegraphics[height=1.5cm]{images/usp.png}
}
\lhead{%
  \includegraphics[height=1.5cm]{images/iag.png}
}
\cfoot{%
  \HeaderFont{}
  \Centering
  \onehalfspacing
  Rua do Matão, 1226 - São Paulo, SP, Brasil, 05508-090
  \newline
  Tel: \Telefone{};
  E-mail: \href{mailto:\Email}{\Email};
  Website: \href{https://www.iag.usp.br}{www.iag.usp.br}
}
\renewcommand{\headrulewidth}{0pt}
\renewcommand{\footrulewidth}{1pt}
\preto{\footrule}{\color{lightgray}}

%%%%%%%%%%%%%%%%%%%%%%%%%%%%%%%%%%%%%%%%%%%%%%%%%%%%%%%%%%%%%%%%%%%%%%%%%%%%%%%
\begin{document}

\begin{flushleft}
  Programa de Embaixadores 2024-2025
  \\
  Rede Brasileira de Reprodutibilidade
\end{flushleft}
\begin{flushright}
  \today
\end{flushright}
\Padding

\noindent
Re: Carta de Motivação
\Padding

% Porque se interessou pelo tema e envolvimento prévio com atividades
% relacionadas a ele

Meu primeiro contato com o movimento de \href{https://www.fsf.org/}{software
livre} foi durante meu curso de bacharelado em geofísica na Universidade de São
Paulo,
onde utilizávamos computadores com o sistema GNU/Linux e os softwares
\href{https://en.wikipedia.org/wiki/Seismic_Unix}{Seismic Unix}
e \href{https://www.generic-mapping-tools.org/}{Generic Mapping Tools} (GMT)
em nossas aulas.
Fui cativado pelos ideais do movimento de garantir a todos a liberdade para
modificar e experimentar com programas e a cultura de se desenvolver produtos
para o bem comum de maneira colaborativa e transparente.
Para mim, esses são os ideais fundamentais para a ciência.

Desde a elaboração de meu primeiro artigo
\href{https://github.com/pinga-lab/paper-planting-densities}{(Uieda \& Barbosa
(2012)}, decidi que iria sempre buscar atingir esses ideais de transparência e
colaboração sem barreiras em tudo o que faço.
Publiquei o \href{https://doi.org/10.6084/m9.figshare.91574}{material
suplementar do artigo} na plataforma
\href{https://figshare.com/authors/Leonardo_Uieda/97471}{figshare}
e todo o código utilizado para produzir os resultados e figuras do artigo estão
disponíveis no repositório do GitHub \href{https://github.com/pinga-lab/paper-planting-densities}{pinga-lab/paper-planting-densities}.
Durante a pós-graduação, ministrei para o grupo de alunos uma
\href{https://doi.org/10.6084/m9.figshare.774537.v3}{palestra e workshop sobre
práticas de ciência aberta}, incentivado meus contemporâneos a publicarem seus
dados e códigos para que os resultados de análises e modelagem numérica
pudessem ser reproduzidos.
Mais recentemente, realizei uma
\href{https://www.leouieda.com/2022-05-06-spin-open-science/}{apresentação
semelhante para um programa de pós-graduação internacional}.

Atualmente todos os artigos produzidos pelo meu grupo de pesquisa
(o \href{https://www.compgeolab.org/}{Computer-Oriented Geoscience Lab}),
e diversas colaborações, incluem todo o código e dados necessários para
reproduzir todos os resultados apresentados.
Mais que isso, busco utilizar somente dados que estão disponíveis com licenças
abertas (e.g., \href{https://creativecommons.org/licenses/by/4.0/}{CC-BY})
e publicar em acesso aberto para garantir que qualquer pessoa interessada
possa reproduzir meus resultados, seja em revistas de acesso livre ou
publicando preprints em repositórios da área.
Esses princípios estão descritos de forma mais extensa no
\href{https://www.compgeolab.org/manual/}{manual de operações do CompGeoLab},
um documento que criamos para informar novos colaboradores e membros
do grupo sobre nossas expectativas em relação à ciência aberta e
reprodutibilidade.

Durante o tempo que passei na Inglaterra, acompanhei as atividades do UKRN com
grande interesse por conta  da minha participação no
\href{https://www.leouieda.com/blog/ssi-fellowship.html}{Software
Sustainability Institute} e fiquei muito feliz ao descobrir a Rede Brasileira
de Reprodutibilidade.
Acredito que organizações independentes como a RBR, que reúne pessoas motivadas
por um interesse em comum, são a chave para realizar as mudanças estruturais
que necessitamos na ciência, nos processos de financiamento, formação de
pesquisadores e contratação.
Meu maior interesse no programa de embaixadores é me conectar a essa comunidade
brasileira e interdisciplinar de ciência aberta e reprodutibilidade.


\Padding

\begin{flushleft}
  Atenciosamente,

  \Padding
  \Nome{}
  \\[0.25cm]
  {
    \color{mediumgray}
    \small
    \Cargo
    \\
    Departamento de Geofísica
    \\
    Instituto de Astronomia, Geofísica e Ciências Atmosféricas
    \\
    Universidade de São Paulo
    \\
    Website: \url{https://www.leouieda.com}
    \\
    Grupo de pesquisa: \url{https://www.compgeolab.org}
  }
\end{flushleft}
\end{document}
