% Template for a letter on IAG-USP letterhead

%%%%%%%%%%%%%%%%%%%%%%%%%%%%%%%%%%%%%%%%%%%%%%%%%%%%%%%%%%%%%%%%%%%%%%%%%%%%%%%
% Set a class and general configuration
\documentclass[a4paper,onecolumn,10pt]{article}

%%%%%%%%%%%%%%%%%%%%%%%%%%%%%%%%%%%%%%%%%%%%%%%%%%%%%%%%%%%%%%%%%%%%%%%%%%%%%%%
% Set variables with the title, authors, etc.
\newcommand{\Titulo}{Carta de motivação - Embaixadores da Rede Brasileira de Reprodutibilidade}
\newcommand{\Nome}{Leonardo Uieda}
\newcommand{\Cargo}{Professor Doutor}
\newcommand{\Email}{uieda@usp.br}
\newcommand{\Telefone}{+55 11 30914787}

%%%%%%%%%%%%%%%%%%%%%%%%%%%%%%%%%%%%%%%%%%%%%%%%%%%%%%%%%%%%%%%%%%%%%%%%%%%%%%%
% Import the required packages
\usepackage[utf8]{inputenc}
\usepackage[TU]{fontenc}
\usepackage[brazil]{babel}
%\usepackage[english]{babel}
\usepackage{graphicx}
\usepackage{hyperref}
\usepackage{fancyhdr}
\usepackage{geometry}
\usepackage{microtype}
\usepackage{xcolor}
% improved urls with proper hyphenation
\usepackage{xurl}
% Use a different font
\usepackage[default,scale=0.95]{opensans}
% Icons and fonts (requires using xelatex or luatex)
\usepackage{fontawesome5}
\usepackage{academicons}
% Control the font size
\usepackage{anyfontsize}
\usepackage{setspace}
% Generate random text
\usepackage{lipsum}
% Better left and right align
\usepackage{ragged2e}

%%%%%%%%%%%%%%%%%%%%%%%%%%%%%%%%%%%%%%%%%%%%%%%%%%%%%%%%%%%%%%%%%%%%%%%%%%%%%%%
% Configuration of the document

\geometry{%
  left=25mm,
  right=15mm,
  top=15mm,
  bottom=15mm,
  headsep=15mm,
  headheight=15mm,
  footskip=7mm,
  includehead=true,
  includefoot=true
}

% Control line spacing
\onehalfspacing
\newcommand{\Padding}{\vspace{0.5cm}}

% Custom colors
\definecolor{darkgray}{gray}{0.4}
\definecolor{mediumgray}{gray}{0.5}
\definecolor{lightgray}{gray}{0.9}
\definecolor{mediumblue}{HTML}{2060c2}
\definecolor{lightblue}{HTML}{f7faff}

% Make urls use the same font as every other text
\urlstyle{same}

% Configure hyperref and add PDF metadata
\hypersetup{
    colorlinks,
    allcolors=mediumblue,
    pdftitle={\Titulo},
    pdfauthor={\Nome},
    breaklinks=true,
}

% Configure header and footer
\newcommand{\HeaderFont}{\footnotesize\color{mediumgray}}
\pagestyle{fancy}
\fancyhf{}
\rhead{%
  \includegraphics[height=1.5cm]{images/usp.png}
}
\lhead{%
  \includegraphics[height=1.5cm]{images/iag.png}
}
\cfoot{%
  \HeaderFont{}
  \Centering
  \onehalfspacing
  Rua do Matão, 1226 - São Paulo, SP, Brasil, 05508-090
  \newline
  Tel: \Telefone{};
  E-mail: \href{mailto:\Email}{\Email};
  Website: \href{https://www.iag.usp.br}{www.iag.usp.br}
}
\renewcommand{\headrulewidth}{0pt}
\renewcommand{\footrulewidth}{1pt}
\preto{\footrule}{\color{lightgray}}

%%%%%%%%%%%%%%%%%%%%%%%%%%%%%%%%%%%%%%%%%%%%%%%%%%%%%%%%%%%%%%%%%%%%%%%%%%%%%%%
\begin{document}

\begin{flushleft}
  Programa de Embaixadores 2024-2025
  \\
  Rede Brasileira de Reprodutibilidade
\end{flushleft}
\begin{flushright}
  \today
\end{flushright}
\Padding

\noindent
Re: Carta de Motivação
\Padding

% Porque se interessou pelo tema e envolvimento prévio com atividades
% relacionadas a ele

Meu nome é Leonardo Uieda, sou geofísico de formação e professor na
Universidade de São Paulo.
Minha pesquisa faz uso de pequenas perturbações nos campos de gravidade e magnético da Terra para investigar o interior do planeta.
As aplicações variam de escala global, como \href{https://github.com/pinga-lab/paper-moho-inversion-tesseroids}{determinar a espessura da crosta terrestre},
até escala microscópica, como
\href{https://github.com/compgeolab/micromag-euler-dipole}{determinar as
propriedades magnéticas de pequenos grãos minerais a partir de dados de
microscopia}.
O uso e o desenvolvimento de software livre permeia todas as minhas atividades
de pesquisa e ensino, principalmente através do projeto
\href{https://www.fatiando.org/}{Fatiando a Terra}.
Sou um entusiasta por ciência aberta
\href{https://github.com/leouieda/agu2010}{desde meu primeiro contato com a pesquisa} e
busco
\href{https://meetingorganizer.copernicus.org/EGU21/session/40092}{divulgar a prática da ciência aberta} amplamente em minha disciplina.

Meu primeiro contato com o movimento de \href{https://www.fsf.org/}{software
livre} foi durante meu curso de bacharelado em geofísica na Universidade de São
Paulo,
onde utilizávamos em nossas aulas computadores com o sistema GNU/Linux e os
softwares livres
\href{https://en.wikipedia.org/wiki/Seismic_Unix}{Seismic Unix}
e \href{https://www.generic-mapping-tools.org/}{Generic Mapping Tools},
que são projetos de referência na geofísica.
Fui cativado pelos ideais do movimento, como garantir a todos a liberdade de
acessar e modificar os programas que utilizamos e a cultura do movimento de
desenvolver produtos para o bem comum de maneira colaborativa e transparente.
Para mim, esses também são os ideais fundamentais que representam a ciência.
Esse interesse em software livre me levou naturalmente à discussão sobre
ciência aberta e reprodutibilidade ainda durante minha graduação e continuando
durante a pós-graduação.

Desde a elaboração de meu primeiro artigo
(\href{https://github.com/pinga-lab/paper-planting-densities}{Uieda \& Barbosa,
2012}), decidi que iria sempre buscar atingir esses ideais de transparência e
colaboração sem barreiras em tudo o que faço.
Publiquei o
\href{https://doi.org/10.6084/m9.figshare.91574}{material suplementar do
artigo} na plataforma
\href{https://figshare.com/authors/Leonardo_Uieda/97471}{figshare},
que havia sido inaugurada no ano anterior.
Também desenvolvi todo o código utilizado para produzir os resultados e figuras
do artigo em um \href{https://github.com/pinga-lab/paper-planting-densities}{repositório na plataforma GitHub}, que foi publicado junto com o
artigo.
Durante a pós-graduação, ministrei para meus contemporâneos uma
\href{https://doi.org/10.6084/m9.figshare.774537.v3}{palestra e
workshop sobre práticas de ciência aberta},
incentivado-os a publicar seus dados e códigos para que os resultados de
análises e modelagem numérica pudessem ser reproduzidos.
Mais recentemente, realizei uma
\href{https://www.leouieda.com/2022-05-06-spin-open-science/}{apresentação
semelhante para um programa de pós-graduação internacional}.

Atualmente todos os artigos produzidos pelo meu grupo de pesquisa
(o \href{https://www.compgeolab.org/}{Computer-Oriented Geoscience Lab})
incluem todo o código e dados necessários para reproduzir todos os resultados
apresentados.
Nossa estratégia é organizar toda a produção dos resultados e a escrita do
artigo em \href{https://github.com/compgeolab/micromag-euler-dipole}{repositórios no GitHub}.
No momento de submissão a uma revista, disponibilizamos o repositório
publicamente e arquivamos o conteúdo do repositórios em plataformas de dados
abertos,
como \href{https://doi.org/10.5281/zenodo.16191}{Zenodo} ou
\href{https://doi.org/10.6084/m9.figshare.22672978.v3}{figshare}.
Mais que isso, buscamos utilizar somente dados que estão disponíveis com
licenças abertas (e.g.,
\href{https://creativecommons.org/licenses/by/4.0/}{CC-BY})
e publicar em acesso aberto para garantir que qualquer pessoa interessada
possa reproduzir meus resultados, seja em revistas de acesso livre ou
publicando \href{https://doi.org/10.31223/X58G7C}{preprints em repositórios da área}.
Esses princípios estão descritos de forma mais extensa no
\href{https://www.compgeolab.org/manual/}{manual de operações do CompGeoLab},
um documento que criamos para informar novos colaboradores e membros
do grupo sobre nossas expectativas em relação à ciência aberta e
reprodutibilidade.

Durante o tempo que passei na Inglaterra, descobri o UKRN por conta  da minha
participação no
\href{https://www.leouieda.com/blog/ssi-fellowship.html}{Software
Sustainability Institute}.
Porém, somente acompanhei as atividades do grupo e não tive a oportunidade de
me envolver mais diretamente.
Por isso, fiquei muito feliz ao descobrir a Rede Brasileira de Reprodutibilidade
quando retornei ao Brasil em 2023.
Acredito que organizações independentes como a RBR, que reúne pessoas
extremamente motivadas por um interesse em comum, são a chave para realizar as
mudanças estruturais que necessitamos na ciência, nos processos de
financiamento, formação de pesquisadores e contratação.
Meu maior interesse no programa de embaixadores é me conectar a essa comunidade
brasileira e interdisciplinar de ciência aberta e reprodutibilidade.
Com isso, espero poder trazer para a geofísica, e para as ciências da Terra em
geral, essa importante discussão sobre reprodutibilidade e ciência aberta.
Algumas áreas da geofísica estão à frente dessas questões, mas o conhecimento
das boas práticas de reprodutibilidade e ciência aberta não estão igualmente
difundidas, principalmente na pós-graduação.
Sinto falta dessas discussões no meu instituto e vejo nas alunas e alunos da
pós-graduação um interesse enorme pelo tema, que não é abordado fora de breves
conversas no corredor ou ocasionalmente em um café da tarde.
Também espero contribuir para as atividades da RBR, trazendo experiência na
área de reprodutibilidade computacional, geociências, software livre e ciência
de dados.


\Padding

\begin{flushleft}
  Atenciosamente,

  \Padding
  \Nome{}
  \\[0.25cm]
  {
    \color{mediumgray}
    \small
    \Cargo
    \\
    Departamento de Geofísica
    \\
    Instituto de Astronomia, Geofísica e Ciências Atmosféricas
    \\
    Universidade de São Paulo
    \\
    Website: \url{https://www.leouieda.com}
    \\
    Grupo de pesquisa: \url{https://www.compgeolab.org}
  }
\end{flushleft}

\Padding
\Padding
{
  \color{mediumgray}
  \noindent
  PS: Esta carta pode ser reproduzida a partir do código fonte \LaTeX{}
  disponível em \url{https://github.com/leouieda/rbr-embaixador} (CC-BY).
}
\end{document}
