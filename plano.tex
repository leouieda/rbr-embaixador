% Template for a letter on IAG-USP letterhead

%%%%%%%%%%%%%%%%%%%%%%%%%%%%%%%%%%%%%%%%%%%%%%%%%%%%%%%%%%%%%%%%%%%%%%%%%%%%%%%
% Set a class and general configuration
\documentclass[a4paper,onecolumn,10pt]{article}

%%%%%%%%%%%%%%%%%%%%%%%%%%%%%%%%%%%%%%%%%%%%%%%%%%%%%%%%%%%%%%%%%%%%%%%%%%%%%%%
% Set variables with the title, authors, etc.
\newcommand{\Titulo}{Plano de atuação - Embaixadores da Rede Brasileira de Reprodutibilidade}
\newcommand{\Nome}{Leonardo Uieda}
\newcommand{\Cargo}{Professor Doutor}
\newcommand{\Email}{uieda@usp.br}
\newcommand{\Telefone}{+55 11 30914787}

%%%%%%%%%%%%%%%%%%%%%%%%%%%%%%%%%%%%%%%%%%%%%%%%%%%%%%%%%%%%%%%%%%%%%%%%%%%%%%%
% Import the required packages
\usepackage[utf8]{inputenc}
\usepackage[TU]{fontenc}
\usepackage[brazil]{babel}
%\usepackage[english]{babel}
\usepackage{graphicx}
\usepackage{hyperref}
\usepackage{fancyhdr}
\usepackage{geometry}
\usepackage{microtype}
\usepackage{xcolor}
% improved urls with proper hyphenation
\usepackage{xurl}
% Use a different font
\usepackage[default,scale=0.95]{opensans}
% Icons and fonts (requires using xelatex or luatex)
\usepackage{fontawesome5}
\usepackage{academicons}
% Control the font size
\usepackage{anyfontsize}
\usepackage{setspace}
% Generate random text
\usepackage{lipsum}
% Better left and right align
\usepackage{ragged2e}

%%%%%%%%%%%%%%%%%%%%%%%%%%%%%%%%%%%%%%%%%%%%%%%%%%%%%%%%%%%%%%%%%%%%%%%%%%%%%%%
% Configuration of the document

\geometry{%
  left=25mm,
  right=15mm,
  top=15mm,
  bottom=15mm,
  headsep=15mm,
  headheight=15mm,
  footskip=7mm,
  includehead=true,
  includefoot=true
}

% Control line spacing
\onehalfspacing
\newcommand{\Padding}{\vspace{0.5cm}}

% Custom colors
\definecolor{darkgray}{gray}{0.4}
\definecolor{mediumgray}{gray}{0.5}
\definecolor{lightgray}{gray}{0.9}
\definecolor{mediumblue}{HTML}{2060c2}
\definecolor{lightblue}{HTML}{f7faff}

% Make urls use the same font as every other text
\urlstyle{same}

% Configure hyperref and add PDF metadata
\hypersetup{
    colorlinks,
    allcolors=mediumblue,
    pdftitle={\Titulo},
    pdfauthor={\Nome},
    breaklinks=true,
}

% Configure header and footer
\newcommand{\HeaderFont}{\footnotesize\color{mediumgray}}
\pagestyle{fancy}
\fancyhf{}
\rhead{%
  \includegraphics[height=1.5cm]{images/usp.png}
}
\lhead{%
  \includegraphics[height=1.5cm]{images/iag.png}
}
\cfoot{%
  \HeaderFont{}
  \Centering
  \onehalfspacing
  Rua do Matão, 1226 - São Paulo, SP, Brasil, 05508-090
  \newline
  Tel: \Telefone{};
  E-mail: \href{mailto:\Email}{\Email};
  Website: \href{https://www.iag.usp.br}{www.iag.usp.br}
}
\renewcommand{\headrulewidth}{0pt}
\renewcommand{\footrulewidth}{1pt}
\preto{\footrule}{\color{lightgray}}

%%%%%%%%%%%%%%%%%%%%%%%%%%%%%%%%%%%%%%%%%%%%%%%%%%%%%%%%%%%%%%%%%%%%%%%%%%%%%%%
\begin{document}


\section*{Plano de atividades -- Programa de Embaixadores RBR 2024-2025}

\noindent
\textbf{Candidato:} \href{https://www.leouieda.com}{Leonardo Uieda}
\Padding


\noindent
Algumas das atividades que gostaria de realizar como embaixador da RBR são:

\Padding
\noindent
\textbf{Curso de ferramentas para reprodutibilidade:}
Em Fevereiro de 2025, eu e meus alunos ofereceremos um curso de 15 horas na
XXVII Escola de Verão de Geofísica do IAG - USP intitulado
\href{https://github.com/compgeolab/kit}{``Kit de sobrevivência digital
para cientistas''}. O curso ensinará como utilizar algumas ferramentas de software livre para melhorar a reprodutibilidade
computacional: shell/bash, git, GitHub, make, LaTeX. Nosso objetivo é
tornar os participantes capazes de rodar sua análise de dados, produzir
figuras e gerar um PDF de um artigo, tudo com um único comando. Abordaremos
também o tópico de reprodutibilidade do ambiente computacional e algumas
ferramentas disponíveis na linguagem Python para gerar ambientes reprodutíveis
em máquinas diferentes.
Através do programa de embaixadores, gostaria de expandir o escopo do curso
além da geofísica e ciências da Terra.
Pretendo refinar o material e a abordagem do curso e oferecê-lo novamente
para um público mais amplo.
Uma das formas de expandir o curso, seria pareá-lo com um workshop da
organização \href{https://software-carpentry.org/}{Software Carpentry}, da qual
sou instrutor credenciado.
Também exploraria oferecimento online para outras instituições, no qual tenho
\href{https://www.leouieda.com/cv/#5}{experiência}.
Vejo esse curso como um complemento ao material desenvolvido pelo Software
Carpentry e que poderá ser utilizado em qualquer área da ciência para ensinar
reprodutibilidade computacional.

\Padding
\noindent
\textbf{Grupo de discussão de ciência aberta:}
Atualmente, não existe um grupo de discussões sobre ciência aberta no
Instituto de Astronomia, Geofísica e Ciências Atmosféricas da USP. Eu gostaria
de fundar um, talvez como filial do
\href{https://reproducibilitea.org/}{ReproducibiliTea}, que envolvesse as três
áreas do instituto e também áreas adjacentes, como oceanografia e geologia.
Minha ideia é realizar encontros mensais para discutir artigos,
técnicas e problemas relacionados a ciência aberta e reprodutibilidade. Os
encontros seriam apoiados por uma plataforma de chat (zulip, mattermost,
telegram ou signal) para manter o grupo engajado e organizar a discussão. Em
seguida, procuraria divulgar o grupo para outras instituições de ciências da
Terra para promover a participação nas discussões através da plataforma de chat
e a fundação de mais capítulos locais para encontros presenciais.
Este grupo mais abrangente de diversas instituições poderia se encontrar em
eventos nacionais.
Idealmente, o grupo terá uma participação na RBR e poderá contribuir para
sugestão de políticas para periódicos e agências de fomento nas áreas de
atuação do instituto.

\end{document}
